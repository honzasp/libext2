\chapter{High-level overview}

Of course, my plans failed miserably. Indeed, I could not even find a Rust
library that would allow me to read from a file asynchronously and I should soon
find that the design principles that worked for \Cplusplus{} did not work well in
Rust.  Designing data structures and control flow is much more demanding when
the type system must be assured that everything is OK. When I think about it
now, I should probably try harder, but I quickly decided to throw in the towel
and implement the library first synchronously, and maybe rewrite it later, armed
with the experience gained on the road. This rewrite is now unlikely, so I will
present the synchronous version from now on.

\section{File system structure}

An ext2 filesystem \cite{ext2-layout} begins with a superblock that stores the
crucial information about the filesystem concerning the layout and the details
of the format. The disk is divided into blocks of $2^{10+i}$ bytes for some $i
\geq 0$. The whole volume is split into equally-sized block groups. Every block
group contains a part of the inode table and a bitmap that is used to mark
blocks used in the group. There is also a bitmap that marks whether each inode
from the block group is used or not. To enhance data locality, the data of an
inode is primarily allocated into the block group that stores the inode, and
inodes of directory entries are first placed into the block group of the
directory.

The first 12 blocks of data are referenced directly from the inode. Then there
is a reference to an indirect block, that contains an array of references to
next blocks. Then there is a doubly indirect block, that references another
indirect blocks, and finally a trebly indirect block referencing following
doubly indirect blocks.

The directories are just special files that contain a linked list of entries.
There are various restrictions on this list (for example, no entry can cross
block boundaries, the entries can only point forward, or every entry must start
on a four-byte boundary), but the order or the items is not specified, so
directory lookup is an $O(n)$ process. There is also an extension of the
file system that allows the directories to be stored more efficiently using a
hash map, but I did not implement it.

To make testing the file system easier, I created a simple FUSE wrapper that
allowed me to mount the filesystem and work with it using regular Linux APIs.
However, I did not create any tests.\footnote{Mainly because the cost of errors
in projects like this is so low that testing does not pay for itself.} Honestly,
I would not dare to access any physical hard drive containing valuable data with
my library.

\section{Code organization}

I tried various ways of organizing and subdividing the library. First, I tried
to organize the work into several layers, inspired by \cite{sicp}: the lowest
layer would deal with the raw bytes, the next layer would manage inodes and
inode data blocks and the top layer would provide high-level access to files and
directories. However, the boundaries between the layers were not very clear and
I was afraid of over-abstraction, so I abandoned this model.

Then I attempted to organize the code around the objects. The procedures dealing
with inodes would become methods of the \texttt{Inode} object, the procedures
for working with groups would be associated with \texttt{Group} and so on. This
would work without a little trouble: 90~\% of the code deals with inodes, so I
ended up with one huge module defining the inode with small modules orbiting
around it.

In the end I stuck to the model used in C or in Lisps -- most of the procedures
are standalone functions defined in modules that are loosely grouped around the
actions and data structures. There is a module \texttt{fs} that contains
functions that mount and umount the filesystem, modules \texttt{encode} and
\texttt{decode} for translating between internal and external representations of
various data structures, \texttt{inode} for generic operations on inodes,
\texttt{inode\_data} for reading and writing inode data and so on. Most
\texttt{struct}s are defined in module \texttt{defs}. 

Every module in Rust automatically defines a namespace, but I considered it too
painful to prefix every function with the name of the module that it is defined
it, so I created a module named \texttt{prelude} that reexports (or, as Rust
says, \texttt{pub}licly \texttt{use}s) all names from all modules. However, only
\texttt{pub}lic names are available outside of modules. Also, the structure of
modules is not visible to users, because the library exports only a carefully
selected list of functions (so not every function that is \texttt{pub} is
visible to the user). The fields of structs are also public, so they can be
accessed outside of the modules they are defined in. Unfortunately, that means
that the fields are also visible to the user. The solution would be to define
them with private fields in the root module (submodules in Rust can see private
items); I will maybe do this later.
