\chapter{Motivation}

File system is a format for saving sequences of bytes in a simple hierarchy,
decorated with some metadata, inside a linear address space. These formats are
understood by operating systems and abstracted away to provide uniform model of
files.

Ext2 is a file system developed for Linux. Even though it has been mostly
replaced by its newer revisions ext3 and ext4, it remains essentially backwards
compatible (at least for reading). The main difference between ext2 and ext3 is
that the latter supports journaling, which enhances performance and reliability
by storing updates in a journal and replaying them later, while always keeping
the device in a consistent state.

\section{Non-blocking IO}

Non-blocking IO (also referred to as asynchronous IO) means that a thread
willing to perform an IO operation is not suspended until the operation
finishes, but immediately continues executing. Later, when the operating system
finishes the request, the thread is notified and can handle the results. This is
usually accomplished by having the thread sit in an event-loop, waiting for
events from the OS and dispatching them to registered handlers.

The other approach, blocking IO, is more usual, but has various downsides.
First, while the process is waiting for an IO to happen, it is blocked and
cannot get other things done. This works fine for processes that start by
reading input, then get hot for a while and finish by writing output, but it
would be disastrous for any (soft) real-time process that has to continuously
respond to requests coming from the outside.

The only way to make things happen while your thread is blocked is to spawn
another thread. To do $n$ things at once, you need at least $n$ threads.
However, there are few tools widely available in modern programming environments
as dangerous as a thread. It is nearly impossible for to properly coordinate so
many threads by hand and avoid all race conditions. Moreover, when the $n$
things to be done are just waiting for events, we are paying the non-negligible
cost of a thread only to make it sleep 99.9~\% of its lifetime.

Event-driven programming avoids the thread synchronization hell, because there
is only one thread running at a time, so no race conditions can occur while an
event handler is executing. On the other hand, the sequencing constructs
provided by the programming language can no longer be used to sequence IO. For
example, when we want to load a list of funny pictures, blend them together and
save them to a file using blocking IO, we can use the usual tools of our
programming language:\footnote{In the code, I use \texttt{\textbackslash} to say
$\lambda$, because \LaTeX is, well, complicated.}

\begin{ttcode}
blend_pictures = \(files) {
  pictures = for file in files { read_picture(file) }
  blended = blend_pictures(pictures)
  write_picture(blended, "blended.jpg")
}
\end{ttcode}

Note that the usual tools for composing computations worked well for composing
IO operations. On the other hand, when we use callbacks to achieve the same
effects with non-blocking IO, we end up with something like:\footnote{Of
course, this works only if we have a garbage collector, so that the values
captured in the closures do not get destroyed too early.}

\begin{ttcode}
blend_pictures = \(files, blend_callback) {
  load_pictures = \(files, pictures, load_callback) {
    if empty?(files) {
      load_callback(pictures)
    } else {
      read_picture(head(files), \(picture) {
        load_pictures(tail(files), cons(picture, pictures), load_callback)
      })
    }
  }

  load_pictures(reverse(files), \(pictures) {
    blended = blend_pictures(pictures)
    write_picture(blended, "blended.jpg", blend_callback)
  }
}
\end{ttcode}

It is instructive to see that when imperative constructs fail, the functional
come to the rescue.\footnote{Another instance of a similar phenomenon is
so-called Visitor desing pattern, needed to encode variants in object-oriented
languages -- it is exactly the way how variants are encoded in pure
$\lambda$-calculus. For me, using such a convoluted way to encode something so
obvious feels like I needed to write \texttt{[->+<]} to move a value between
variables :-)} Indeed, help comes from the language that has no imperative
constructs -- Haskell -- in the form of \texttt{IO} monad. When we use a monad
only to capture the results and allow side-effects, we end up with futures: 

\begin{ttcode}
blend_pictures = \(files) {
  future_map(files, \(file) {
    read_picture(file) 
  }).then(\(pictures) {
    blended = blend_pictures(pictures)
    write_picture(blended, "blended.jpg")
  })
}
\end{ttcode}

A future represents a computation whose result will be known in the future.
Futures can be composed to produce other futures using various combinators (like
\texttt{future_map} or \texttt{.then} in our example). They allow us to use
non-blocking IO with little cognitive overhead compared to blocking IO.

I should also mention that there is another approach to doing many things at
once, which avoids the need to distinguish pure and IO-based computations,
showcased in languages like Go or Erlang. The first ingredient is to make
threads cheap by using a lightweight system of green threads (fibers,
goroutines, ...) on top of OS threads, managed by the language's runtime. The
second ingredient is to provide better tools to avoid race conditions. For
example, Erlang forces all values to be immutable, so there are no data races.
The threads in Erlang communicate and share data using messages, which are more
natural and usually easier to get right than complex interactions of mutexes and
condition variables. Go does not restrict mutability, but it also provides a
message-passing primitives for inter-thread\footnote{The threads in Go are
called goroutines, and spawned by keyword \texttt{go}.} communication.

\section{Rust}

Rust \cite{rust-lang} is a systems programming language developed at Mozilla
Research. It guarantees memory safety without garbage collector by statically
checking ownership of values and lifetimes of references. Compared to
\Cplusplus, probably its closest competitor, it is much more modern, featuring
type inference, traits (similar to type classes in Haskell), rich data types and
pattern matching, to name a few.

Rust runtime used to support green threads, but the language evolved towards
smaller runtime and they were removed.  However, the standard library provides
access to OS threads and implements various synchronization primitives,
including message passing. The strong type system can for example guarantee that
values can pass thread boundaries safely, so that no data races can occur, or
that a value protected by a mutex will not be accessed incorrectly.

\section{Personal motivation}

In summer 2015, I was working for a startup named Bileto on a public transport
routing engine. The engine was written in \Cplusplus, because the speed of the
algorithm was crucial. However, the service needed to communicate with many
other servers over the network, so that the size of the code that managed the
communication in fact exceeded the size of the algorithms that did the actual
work. This IO code was originally using blocking IO and many threads, but I
rewrote it to non-blocking using \texttt{libev} and a home-made library of
futures and promises.

I knew and used Rust before for my personal projects, mainly for a compiler that
I developed for my maturita
exam\footnote{\url{https://github.com/honzasp/spiral}}, so I naturally was quite
interested in how Rust could be used to solve similar problems.\footnote{And, of
course, this library was intended to be a part of my great plan to write an OS.
Thanks to Martin Mareš for opening my eyes :-)}
