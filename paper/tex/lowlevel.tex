\chapter{Low-level details}

The central structure is the \texttt{Filesystem}. It stores a copy of the
superblock, copies of all group descriptors and a cache of inodes. When the
superblock or any of the group descriptors needs to be updated, it is first
modified only in memory and marked as dirty. Later, the changes are written to
disk. This saves us a lot of IO, because the superblock and block descriptors
are changed quite often (they store the count of free block or free inodes, for
example).

Inodes are accessed very often, so they are also cached. However, we cannot
afford to store all inodes in memory, so the size of the cache is
limited.\footnote{I need to confess to a sin: the size of the cache is given as
a literal in the function that clears the cache, not as a properly named
constant or a piece of configuration.} The algorithm that decides which inode
will be removed from the cache and which will stay alive is simple: all inodes
are placed in a queue in the order they are read to memory. When there are too
many inodes in memory, the inode from the other side of the queue is removed. To
keep frequently used inodes in memory, an inode is marked as reused when it is
read or written. When the inode from the front of the queue is marked as reused,
it will be enqueued again and not removed, but the mark will not be preserved.
The cache is write-back, the modifications are kept in the cache and written to
disk when the inode is removed from the cache.

The user always references inodes by their number.\footnote{In the code, I used
a convention that \texttt{ino} means inode number, while \texttt{inode} refers
to the \texttt{Inode} structure.}. Most of the code passes around references to
\texttt{Inode}s and does not directly read or write the inode from the
filesystem. However, care must be taken to keep the \texttt{Inode}s in
circulation and inodes on disk (or in cache) in sync. If a piece of code
modified an \texttt{Inode} and another piece of code read an outdated piece of
information from another copy \texttt{Inode}, both referring to the same
physical inode, trouble would happen.

The in-memory structures are not in the same binary format as the structures on
disk, so they need to be encoded and decoded manually. This brings a small
performance penalty and a small decrease in memory consumption (we do not decode
all the fields and include no padding for future extension), but the main reason
is that this approach is safe, because we do not hazard with the memory
guarantees of the language, convenient, mostly because we can use
\texttt{enum}s, and portable, because the data on disk is always stored in
little-endian. There is a caveat, however: because we ignore some of the bytes
when decoding inodes and superblocks, we must first read the corresponding byte
range, overwrite the bytes we care about, and write back. This preserves the
possibly important fields that are currently ignored. The superblock is so
important that the verbatim copy of its bytes is constantly kept in memory.

To avoid any troubles with overflowing numbers, exclusively 64-bit numbers are
used throughout the code. This is not strictly necessary, because most
quantities can be only 32-bit (inode numbers, block numbers, ...), but it saves
us a lot of explicit casting and mitigates the risk of mistakes. Rust also
performs overflow checking at runtime in debug builds, increasing the safety.

Error handling is managed via standard Rust approach, the \texttt{Result} type
and \texttt{try!} macro. \texttt{Result} is a simple algebraic data type with
two variants representing either success or an error (like \texttt{Either} in
Haskell). Explicitly checking every return value would be painful,\footnote{This
is the preferred way of handling errors in Go. This is also the main reason why
I do not really like Go.} while using the monadic combinators (\texttt{.map},
\texttt{.then}) is sometimes convenient, but often it is not. However, one
simple macro can change the situation. \texttt{try!(expr)} evaluates the
\texttt{expr}, which is of type \texttt{Result<T, E>}\footnote{\texttt{T} is the
success, \texttt{E} is the error -- this is different from Haskell.}. If this
was a success, the value of the macro is simply the associated value. On the
other hand, if the result was an error, the macro returns from the current
function with the same error. This exploits the fact that most of the time,
errors should simply be passed up the call stack, but the possible sources of
errors are still explicitly marked. In effect, we gain the comfort of
language-level exceptions without their drawbacks -- error handling is explicit,
requires no language support and creates no corner cases (compare with
\Cplusplus).

In fact, Rust also has a form of implicit exceptions, called thread panics. Code
usually panics when there is a contract violation, for example access out of
array bounds, division by zero or arithmetic overflow (in debug builds). These
panics cause stack unwinding that causes destructors to run, but cannot be
caught, so the whole thread is terminated. This policy forces fault isolation
and tries to avoid most of the problems caused by inconsistent data after an
exception (remember that a thread cannot share data with other threads).
